\chapter{Introduction}
\addcontentsline{toc}{chapter}{Introduction}
%Description of processes: 
%more than just memory, also unit of computation, privilege, execution state etc.
%The only available abstraction at the OS level for memory isolation.
%Problem: it is very expensive.
%TRIALS to decouple memory isolation from process abstraction, e.g., LWC and others. (cite)
%Goal is to have a low overhead while enabling to maintain several execution states.
%+ portability on any linux plateform.
%We inspired ourselves from the lwc implementation, and imported the idea to dune.
%Advantage of working with dune is:
%We define our own interface and have control over everything
%We don't have to respect all the linux implementations etc.
%Enables to have our own VM implementation, a simplified version of linux.
%Can tune it to our needs (e.g., managing memory regions etc.)
%Chapter 2 is related work.
%Chapter 3 is the design and interface.
%Chapter 4 implementation.
%Chapter 5 future work
%Chapter 6 conclusion.

Processes are the common abstraction in general purpose kernels to represent a
unit of isolation in terms of memory, privilege, and execution state.
Processes enable to isolate applications from each other, by preventing one process
from accessing another's memory, privileged ressources, and state.
However, process management, context switching, and ressource accountability introduce 
non-negligeable overheads at the OS-level, especially compared to the associated hardware costs.

While the thread abstraction enables to isolate units of computation,
memory isolation is attainable only by going through the costly creation and management 
of a new process.
Others[CITE] have identified this discrepancy and designed new interfaces, henceforth decoupling
memory isolation, priveleges, and execution states.
A process may then contain several \textit{contexts}, each with their own privileges,
file descriptors, credentials and virtual memory mappings, yielding isolation at a low overhead while
maintaining code efficiency and safety.
%Maybe a little more about contexts, orthogonal to threads etc.

Unfortunately, existing implementations rely on a modified kernel, which in turn 
creates maintainability as well as forward and backward compatibility issues.
For example, \textit{Light-Weight Contexts}[CITE] are implemented in the FreeBSD 11.0 kernel.
In order to use them, one would have to compile and install the modified kernel,
with no guarantee that future kernel updates and patches will be retrofitted into it.
The mentionned issues impede wide-spread adoption and, subsequently, evolution of these solutions.
On the other hand, abstractions such as \textit{lwc} require access to kernel level ressources 
and are hardly decoupled from it.
As they are not part of the usual programmer's utility-belt, forcing them into mainstream 
kernels might be hard, even more so without users support.
This therefore seems to be a circular problem.

We believe that providing the context abstraction via efficient virtualization would enable
to solve the aforementioned limitations.
We implement \textit{Light-Wight Contexts} as a Dune[CITE] library.
%Give high-level description of dune here. Detailed one will be in the related work part.
Dune provides non-intrusive compatibility with the 64-bit x86 Linux kernel, with direct access
to hardware features such as ring protection, page tables, and tagged TLBs.
We implement virtual memory and file descriptors management with low overheads and flexible \textit{lwc} compatible interfaces.
%Emphasise the fact that we can have a simpler mm mgmt than linux, hence making it cheaper to create/share/destroy memory regions.

This paper's contributions are: 
%TODO:  

The rest of this paper is organized as follows: we present related work in Section[REF], describe 
our design and interface for lwcs in Section[REF]. Section[REF] contains modifications done to Dune for lwc support.
Evaluation and results for our implementation of lwcs are detailed in Section[REF].
Section[REF] summarizes future work and we conclude in Section[REF].


