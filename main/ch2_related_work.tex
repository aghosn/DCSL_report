\chapter{Related Work}
%Wedge
%Description, kind of what we want except that it still relies on a mapping 
%between the memory region and an "execution thread", i.e., sthreads are associated
%with some function to run. Not good, we want more separation.
%LWC
%Better abstraction, description and don't forget to tell how it was integrated
%inside a kernel, i.e., meaning it had to modify the kernel and respect lots of APIs.
%Dune
%Description of dune (vtx etc.), enables to simplify the interface we consider by
%managing the memory ourselves. Hence easier to implement LWC there.
%"Like implementing a small simple kernel", low impact on performances.
SOMETHING TO SAY WE'LL present both contexts solutions and virtualization.

\textit{Wedge}[CITE] relies on Linux processes to implement \textit{sthreads} and provide 
privilege separation and memory isolation.
\textit{Sthread}s are default-deny compartments within a single address space.
A compartement is defined by a thread of control and an associated security policy specifying
memory access capabilities, file descriptors, privileges, and \textit{callgates} accesses, i.e., computation subroutines with different privileges.
%Sthreads are a variant of Linux processes, where the inherited memory regions and file descriptors
%can be specified by a security policy.
By default, a newly created shtread only has a copy-on-write access to a pristine snapshot of the parent process's memory, taken just before the execution of its \textit{main} function.
[CITE AUTHORS] argue that this pristine snapshot does not contain any sensitive data that could be leaked
to the child sthread.
Any sthread can be reverted to this initial state.
Wedge also provides an analysis tool meant to help programmers refactor their existing applications into isolated compartements.
%comment about the fact that this is just a help, not automated and hence still need to invest into it?

\textit{Light-weight Contexts}[CITE] provide independent units of protection, privilege, and execution state within a process.
\textit{LwC}s are orthogonal to threads, i.e., lwCs are non-schedulable entities, and hence do not incur a scheduling cost while switching between contexts.
By default, an lwC inherit a private copy of the calling context's state at the time of the call, including
per-thread register values, virtual memory, file descriptors and credentials.
The parent lwC has a fine-grained control over ressources visible and shared with the child lwC, at creation time.
Unlike Wedge[CITE], lwCs are able to snapshot and resume execution in any state, therefore making it more flexible.
LwC's rely on a modified version of the FreeBSD 11.0 kernel.


Dune[CITE] is a 64 bit x86 Linux kernel module and user-level library that relies on virtualization hardware to provide a process abstraction.
Dune[CITE] allows direct access to hardware features such as ring protection, page tables, and tagged TLB's, while preserving existing operating system interfaces for processes.
Dune deviates from standard VMMs by exposing a \textit{process} environment instead of a machine environment.
While not able to support a regular guest OS, Dune proves to be ligther and more flexible.


process == narrower interface -> state of Dune is simpler to solve
hardware either directly accessible or fall back to linux

