\chapter{Implementation}
%Redoing the memory management:
%pages, pgroot, vmas, mms. New hierarchy, backward compatibility preserved though
%all the code is available (hide mms behind a flag).
%
%Sandbox is backported to dune.
%Describe what the sandbox is and the different challenges I faced.
%
%Now we have lwc as a library, inside a program ran inside the sandbox.
%
%lwc implementation details, tricks, e.g., the mapping between host and guest,
%translating the syscall arguments (i.e., the gvas to hvas)

%TODO introduction for the chapter.

\section{Multiple Address Spaces in Dune}
%The structural interfaces.
%Replacing Dune's initialization and call hierarchy.
%backward compatibility, enable to hide behind a flag.
The lwC library strives to provide multiple address spaces via an abstraction completely decoupled from the process and thread ones.
By default, Dune's handles memory mappings via the use of page tables.
While enabling a fine-grained memory management, page tables prove to be a cumbersome abstraction when performing operations that impact large memory regions.
As explained in [REF], we solved this issue by implementing a simplified version of Linux MMU.
The memory hierarchy is divided into 3 layers of abstraction: memory mapping regions, virtual memory areas, and page tables.

%mms
A memory mapping region (\emph{mm}) represents an address space.
It contains a collection of virtual memory areas, implemented.
Dune's \emph{mm} structure is exposed in Figure [REF].
Our implementation is an over-simplification compared to Linux's \emph{mms}.
In Linux, the virtual memory areas are handled as both a red-black tree and a double-linked list, which enables  efficient address range search.
In our case, for the moment, we only have a double-linked list for \emph{vmas}, i.e., the \emph{mmap} attribute.
Simplifying this aspect of the MMU's implementation enabled to have a working and stable prototype quickly.
A tree-based representation is a handy optimization that we will add to our implementation.\\
An \emph{mm} keeps a reference to the page table hierarchy corresponding to its mapping.
This reference is a simple pointer to a \emph{pent\_t} structure.\\
While Linux's \emph{mm} contains several structure variables used for management and statistics by the OS, Dune's implementation is much simpler.
The \emph{mm} are part of a global queue that resides in kernel space and contains all the memory mappings created.
A direct implication is that creating, initializing, using, and managing \emph{mm}s in Dune requires less operations than in Linux.

%vmas
A virtual memory area (\emph{vma}) is an abstraction representing a continuous virtual memory portion accessed in a uniform fashion, i.e., a virtual memory segment with certain access rights.
Dune's \emph{vma} representation is described in Figure [REF].
Once again, the implementation is greatly simplified compared to Linux, which requires more management variables and provides more functionalities than we do.\\
A \emph{vma} contains a \emph{vm start} and a \emph{vm end} that define the range of addresses for this area.
Associated access permissions are encapsulated within the \emph{vm flags}, i.e., a high level representation that maps onto the page table bit-flags.
Virtual memory areas keep a reference to the \emph{mm} they belong to.
While introducing a (relatively light) dependency between \emph{mm}s and \emph{vma}s implementations, the \emph{mm} attribute is used as an argument when an operation at the \emph{vma} level needs to propagate changes to the upper level.
In other words, the \emph{mm} structure is never accessed within the \emph{vma} layer, but rather used as an interface to call \emph{mm} routines when needed.

%page tables.
Dune's original page table representation was preserved and extended.
Preserving the original interface and basic implementation was vital in order to retain backward compatibility and support for existing Dune applications.
As explained previously, it is also needed in order to evaluate delays introduced by new levels of indirection in the memory management unit.
The main effort was put on refactoring the code's layout so as to provide better modularity and encapsulate all memory related operations within the new MMU.
The major modifications were applied to Dune's initialization phase and, hence, do not impact runtime performances.

\section{One Address Space fits them all}
%The synchronization
%The system call interposition with syscallmap

\section{LwCs System Call Wrappers}
%saving arguments is not done the way it needs to be done.
%By default dune does not save all registers etc.
%Created a new wrapper.
