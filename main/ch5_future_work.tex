\chapter{Future Work}

%Extend the library, have the system call interposition mechanism, better filtering for file descriptors
%Evaluate delays against dune's original
%Implement openssl as base application
%Improve mmu structure and implementation to improve performances.


%Performance wide
\section{Performance evaluation and improvement}
Applications rely on Dune for performance.
For example, IX[REF] uses access to hardware features in order to provide BLABLA.
As a result, evaluating the new Dune memory management unit needs to be evaluated.
The goal is to measure the overhead introduced by the extra levels of indirection, and reduce them if possible.
As explained, maintaining backward compatibility plays an essential role in the performance evaluation of the new implementation.
If the delays introduced are not tolerable, the user needs to be able to deactivate the mmu hierarchy and fall back to the default page table handling of memory.\\
The current implementation presents opportunities for optimizations.
For example, a memory region keeps track of its virtual memory areas using a double linked list.
In order to speed-up address range search, a search-tree-based representation would be more efficient.
Another example of potential room for optimization is how a address space is copied.
For the moment, the entire memory hierarchy is copied and only the physical pages are marked copy-on-write.
A possible improvement would be to use a similar mechanism at higher levels to reduce the number of allocations and initializations when copying an address space.

\section{Extending the Library}
Dune's lwC library provides the basic lwC functionalities.
While enough to provide memory isolation within a process, the library could be extended to provide an interface for system call interposition.
Dune provide such mechanism at the kernel level.
It would therefore be interesting to allow the lwC users to tap into them and limit the visibility of system calls within a child context.
This functionality is already present within the original lwC library.

\section{Use-cases}
Real-life use cases should be constructed to validated our implementation.
OpenSSL would be a good target application.
The lwC library could be used to enforce isolation between different sessions.