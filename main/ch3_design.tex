\chapter{Design}

\section{Re-Designing Memory Management}
%TODO aghosn: ADD FIGURES!

Light-weight contexts live at the boundary between user applications and the kernel memory management.
As such, they require flexible and efficient high-level memory abstractions.
More precisely, the LwC library needs to copy, add, share, and remove memory mappings between different contexts.
To facilitate these manipulations, an address space needs to be represented as a set of contiguous memory regions with the same access rights, i.e., contiguous portions of the address space that can be treated in the same fashion.

Dune's original design only provides a rigid API for page table manipulations.
Page tables are a low-level memory representation where page-sized entries are accessed via four levels of indirection.
Transformations are performed on a per-page basis.
In the context of Dune, this representation is an over-simplification of memory management that provides the basic functionalities needed by Dune's applications.
In Dune, guest memory is mapped, via the EPT, as closely as possible to the host address space.
This rigid mapping simplifies VMExits by removing the need for syscall argument interposition and address translation.
While acceptable from Dune's point of view, this design is not flexible enough to efficiently implement LwCs.
Moreover, the page-table abstraction is agnostic to memory regions and too low-level for users to easily reflect upon copied and shared memory between contexts.
We thus identified the need for higher-level memory abstractions and designed a new Dune's \textit{mmu}.

Our memory design is a simplified version of the Linux kernel mmu.
We maintain the same layers, i.e., memory regions, virtual memory regions, and page tables.
A memory region (\textit{mm}) represents an address space.
It can be seen as a collection of virtual memory regions (\textit{vmas}).
A vma abstracts a contiguous virtual memory region with uniform access rights.
An mm is associated with a page root, i.e., a hierarchy of page tables, that reflects the vmas it contains.
We enforce modularity by exposing, at each layer, an API with the expected functionalities.
Any layer's implementation can therefore be modified independently.
Any level-specific functionality is completely encapsulated at the relevant layer.
For example, mms and vmas are oblivious to copy-on-write mechanisms implemented at the page table level.
Considering LwCs, mms provide the proper high-level abstraction to perform memory layout operations.
The LwC library can easily copy and modify memory mappings.
All modifications are automatically applied to lower-levels thanks to cascading calls to underlying abstractions interfaces.

The memory hierarchy's built-in modularity allows to tune any memory management level to suit our need.
As long as interfaces are maintained, the implementation's details can be modified as needed.
This extra flexibility, compared to Dune's default mmu implementation, unfolds two major advantages: 1) memory management operations and data-structures can be optimized on a per-application performance basis, and 2) backward compatibility with existing Dune's applications is maintained.
The second point is crucial to evaluate our implementation in Chapter[REF].
With regard to LwC, the first point enables to experiment different mms and vmas implementations.
While only superficially explored in this paper, we plan to perform more extensive experiments to improve LwC performances in the future.

\section{LwC as a Dune Library}
Light-Weight Contexts are implemented as a Dune user library.
We adapted the original LwC design to Dune.
Implementing LwCs in Dune presents major benefits.
First, interfaces are narrower and simpler, especially compared to a standard kernel.
Mmu data-structures are simpler and less expensive to initialize and maintain. 
Second, as explained in Section[REF], memory management's implementation and interfaces can be tuned for LwC operations.
Moreover, underlying data structures can be tailored to simplify LwC common operations and decrease their initialization and execution costs.
%TODO aghosn: develop

%LwC == memory stuff + file descriptors (todo)
%API
%very close to what we get in the original implementation.
%Simpler implementation see section 4.
%root LwC. add table for function calls.
LwCs encapsulate privileges and execution states within a process.
They are independent units of isolation, containing their own virtual address space and file descriptors.
As explained in [REF], LwCs are orthogonal to threads, i.e., a thread can switch back and forth between several lwCs.
As in the original design, we create an \textit{lwC root} when a Dune process is initialized.
This contexts maps the original virtual memory address allocated to the process.\\
Table [REF] presents our lwC API.
We performed slight simplifications to the original API[REF], but kept the same basic functionalities.

%Same:
%lwcCreate. => add figure for structures
The \textit{lwc\_create} operation enables to create a new contexts.
By default, all user space memory mappings are shared between the parent and child contexts using copy-on-write.
The child and parent contexts keep a reference to each other.\\
The \emph{lwc\_create} call does not trigger a switch to the newly created context.
Instead, the caller receives a positive integer as a function call return value.
The \emph{lwc\_res\_t} argument is filled by the system call with a reference to the newly created context, called \emph{n\_lwc}, and a null \emph{caller} attribute.\\
The child, on the other hand, gets '0' as return value, null as \emph{n\_lwc}, and a reference to the parent lwC as \emph{caller} by default.

In the absence of modifiers, the child lwC inherits state at the time of the call, a copy-on-write copy of the user space virtual memory, file descriptors, and credentials from the calling lwC.
The parent context can, however, modify its resources visibility by providing modifiers via the \emph{lwc\_rsrc\_spec} argument.
Entire memory regions can easily be shared or unmapped.
Figure [REF] details the \emph{lwc\_rsrc\_spec} structure.\\
Unlike the original lwC implementation[REF], the saved state only contains the current thread of execution.
This is a simplification step we took for our first lwC implementation.
Saving all thread states requires barrier synchronization upon \emph{lwc\_create} calls, to ensure that all threads are in a consistent state when the snapshot is taken.
This introduces an extra level of complexity that we do not need, for the moment, to benefit from basic lwC functionalities.
Enabling multi-thread support is, however, an important future goal.
%a bit more on the other modifiers that can be specified.

%lwc_switch
A thread can switch between contexts using the \emph{lwc\_switch} function.
The call takes as arguments a reference to the target lwC, a pointer to extra arguments, as well as a \emph{lwc\_rsrc\_spec} pointer.
The caller's state is saved inside the current lwC, i.e., switching back to the caller lwC will resume execution right after the call to \emph{lwc\_switch}.
When an lwC is \emph{switch to}, the caller's lwC is available within the \emph{lwc\_rsrc\_spec.caller} and extra arguments as \emph{lwc\_rsrc\_spec.args}.

%lwc_free
The lwC library provides a method, called \emph{lwc\_free}, that enables to delete an lwC and free its resources.
All pages that are private to the context's address space are freed, the context is removed from the parent's children list, and all references maintained by the context's children are removed.
Dune's page allocator enables to identify shared pages referenced only by this context and return them to the free pool.


%Differences: 
%lwcs are pointers, not file descriptors => simplified impl and LwCs visibility implemented as part of memory mapping instead of file descriptors. Switching back to original design is not that hard.
%Queue of such lwcs, all visible from the kernel space.
%root lwc maps the original dune mapping. 
%No syscall limitation
Dune's lwC library differs from the original implementation in several aspects.
First, lwCs are data structures referenced using pointers, instead of file descriptors.
This simplification enabled to focus on implementing and testing the lwC functionalities, rather than on their management.
In the future, we might revert back to the original design.\\
Another major difference concerns the system call interposition/emulation, using lwCs.
In the original design[REF], an lwC can service system calls on the behalf of one of its children.
This mechanism enables to perform privileged operations in a restricted environment.
Although very interesting, we did not identify this feature as an essential functionality and therefore did not implement it yet.
Furthermore, Dune's sandbox and syscall map (see Section[REF]) already provide the same functionality.
We would thus rather provide an interface within the lwC library that leverages these mechanisms to implement syscall interposition and emulation, than re-implement an existing functionality.